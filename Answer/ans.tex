%----------------Answers-------------%

\Ans{Order of convergence in bisection method is linear.\\
Order of convergence of in newton raphson method is quadratic.}

\Ans{$x^3-x-4=0$
	
	$f(1)=-4$
	
	$f(2)=2$
	
	so, the root lies between 1 and 2
	
	
	$$x_1=1/2(1+2)=1.5$$
	
	
	Step1: $x_1=1.5$, $f(x_1)=-2.125$
	
	Step2: $x_2=1.75$, $f(x_2)=-0.390625$
	
	Step3: $x_3=1.875$, $f(x_3)=0.7167$
	
	Step4: $x_4=1.8125$, $f(x_4)=0.141$
	
	Step5: $x_5=1.78125$, $f(x_5)=-0.129$
	
	Approximate value is 1.78125}

\Ans{Characteristic Equation: If A is a square matrix of order n with 
	elements $a_{ij}$, we can find a column matrix X and a constant 
	$\leftthreetimes$ such that AX=$\leftthreetimes$X or 
	AX-$\leftthreetimes$IX=0.
	On expansion, it gives an $n_{th}$ degree equation in 
	$\leftthreetimes$, called the characteristics equation of the matrix A. 
	Its roots $\leftthreetimes_i$ (i=1,2,...n) are called the eigen values. 
	And corresponding to each eigen value will have a non-zero solution 
	$$X=[x_1, x_2,
	x_3,.....x_n]'$$ 
	which is known as the eigen vector.}


\Ans{Gauss Jordan Method:
	\begin{equation}5x+4y=15\end{equation}
	\begin{equation}
	3x+7y=12
	\end{equation}
	Multiplying $1^{st}$ equation with 3 
	$$15x+12y=45$$ 
	Multiplying $2^{nd}$ equation with 5
	$$15x+35y=60$$
	Subtracting both the equations 
	$$y=\frac{15}{23}=0.6521$$
	Now, putting the value of y in any equation and we get, 
	$$x= \frac{285}{23}=12.391$$}

\Ans{Let, $$x=\sqrt{N}$$
	$$x^2-N=0$$
	$$f(x)=x^2-N$$
	$$f'(x)=2x$$
	By Newton-Raphson,
	$$x_{n+1}=x_n-\frac{f(x_n)}{f'(x_n)}$$
	$$x_{n+1}=x_n-\frac{{x_n}^2-N}{2x_n}$$
	$$x_{n+1}=x_n-\frac{x_n}{2}-\frac{N}{2x_n}$$
	$$x_{n+1}=\frac{x_n}{2}+\frac{N}{2x_n}$$
	Suppose, $$x_n=\frac{A+B}{2}$$
	So, $$x_{n+1}=\frac{A+B}{2*2}+\frac{N}{A+B}=\frac{S}{4}+\frac{N}{S}$$
	$$\therefore S=A+B$$}

\Ans{Gauss Seidel Method:
	$$4x+y+z=4$$
	$$x+4y-2z=4$$
	$$3x+2y-4z=6$$
	Now, 
	\begin{equation}
	x=\frac{1}{4}(4-y-z)
	\end{equation}
	\begin{equation}y=\frac{1}{4}(4+2z-x)\end{equation}
	\begin{equation}z=-\frac{1}{4}(3x+2y-6)\end{equation}
	Put $y=0, z=0$ in (1)
	$$x=1$$
	Put $x=1,z=0$ in (2)
	$$y=\frac{3}{4}=0.75$$
	Put the values of x and y in (3) 
	$$z=\frac{-3}{8}=-0.375$$
	Second iteration:
	
	Put $y=\frac{3}{4}$ and $z=\frac{-3}{8}$ in (1) we get, 
	$$x=\frac{29}{32}=0.90625$$
	put, $x=0.90625$ and $z=-0.375$ in (2) we get, 
	$$y=0.5859$$
	put, $x=0.90625$ and $y=0.5859$ in (3) we get, 
	$$z=-0.5273$$
	Third iteration:
	
	put, $y=0.5859$ and $z=-0.5273$ in (1) we get,
	$$x=0.98535$$
	put, $x=0.98535$ and $z=-0.5273$ in (2) we get, 
	$$y=0.4900$$
	put, $x=0.98535$ and y=$0.4900$ in (3) we get, 
	$$z=-0.5159$$
	Fourth iteration:
	
	put, $y=0.4900$ and $z=-0.5159$ in (1) we get,
	$$x=1.006475$$
	put, $x=1.006475$ and $z=-0.5159$ in (2) we get,
	$$y=0.4904$$
	put, $x=1.006475$ and $y=0.4904$ in (3) we get,
	$$z=-0.4999$$}

%-----------------------------finish----------------------%
